
%**************************************************************
% Acronimi
%**************************************************************
\renewcommand{\acronymname}{Acronimi e abbreviazioni}

\newacronym[description={\glslink{apig}{Application Program Interface}}]
    {api}{API}{Application Program Interface}
    
\newacronym[description={\glslink{html}{HyperText Markup Language}}]
    {html}{HTML}{HyperText Markup Language}

\newacronym[description={\glslink{rest}{Representational State Transfer}}]
    {rest}{REST}{Representational State Transfer}   

\newacronym[description={\glslink{virtualmachine}{Virtual Machine}}]
    {vm}{VM}{Virtual Machine}   


%**************************************************************
% Glossario
%**************************************************************
\renewcommand{\glossaryname}{Glossario}

\newglossaryentry{Cordova}
{
    name=\glslink{Cordova}{Cordova},
    text=Cordova,
    sort=Cordova,
    description={Apache Cordova è un framework per la realizzazione di applicazioni ibride e che offre delle API che permettono di accedere ad alcune funzionalità del dispositivo, come l'accelerometro o la fotocamera}
}

\newglossaryentry{virtual machine}
{
    name=\glslink{virtual machine}{Virtual Machine},
    text=Virtual Machine,
    sort=Virtual Machine,
    description={Software che simula delle risorse hardware e che utilizza queste risorse per eseguire determinate applicazione, in modo che queste possano utilizzare le risorse simulate. Le virtual machine hanno vari utilizzi, in questo caso vengono utilizzate per interpretare il codice JavaScript.}
}

\newglossaryentry{DOM}
{
    name=\glslink{DOM}{DOM},
    text=DOM,
    sort=DOM,
    description={Il DOM o \textit{Document Object Model} è lo standard del W3C per la rappresentazione ad oggetti di documenti strutturati, come le pagine HTML}
}

\newglossaryentry{WebView}
{
    name=\glslink{WebView}{WebView},
    text=WebView,
    sort=WebView,
    description={Componente grafico offerto dalle API native, sia di iOS, sia di Android, che permette la visualizzazione di pagine HTML}
}

\newglossaryentry{rendering}
{
    name=\glslink{rendering}{rendering},
    text=rendering,
    sort=rendering,
    description={Termine inglese che indica l'insieme di attività da svolgere per la rappresentazione grafica di un elemento, nel caso specifico dell'interfaccia grafica di un'applicazione}
}

\newglossaryentry{npm}
{
    name=\glslink{npm}{npm},
    text=npm,
    sort=npm,
    description={Acronimo di Node Package Manager, è un sistema di gestione delle dipendenze per le applicazioni JavaScript che permette di installare librerie di terze parti mediante un'interfaccia a riga di comando}
}

\newglossaryentry{riflessione}
{
    name=\glslink{riflessione}{riflessione},
    text=riflessione,
    sort=riflessione,
    description={In informatica, è la capacità di un programma di analizzare, durante la sua esecuzione, le classi che lo compongono, ricavando così informazioni sulla struttura del proprio codice sorgente}
}
    
\newglossaryentry{gesture}
{
    name=\glslink{gesture}{gesture},
    text=gesture,
    sort=gesture,
    description={Combinazione di movimenti dell'utente effettuati con le dita su un dispositivo touch-screen, che vengono riconosciuti da un'applicazione}
}

\newglossaryentry{tap}
{
    name=\glslink{tap}{tap},
    text=tap,
    sort=tap,
    description={Gesture che consiste in un singolo tocco dello schermo da parte dell'utente, è l'equivalente di un click del mouse}
}

\newglossaryentry{pan}
{
    name=\glslink{pan}{pan},
    text=pan,
    sort=pan,
    description={Gesture che consiste in un tocco prolungato dello schermo da parte dell'utente. Durante l'esecuzione della gesture, l'utente può trascinare il punto di contatto in un modo simile al \textit{drag'n'drop} effettuato con il mouse}
}

\newglossaryentry{Swipe}
{
    name=\glslink{Swipe}{swipe},
    text=swipe,
    sort=swipe,
    description={\`E un particolare tipo di pan, effettuato in modo rapito e in una singola direzione, tipicamente da destra verso sinistra o viceversa}
}

\newglossaryentry{singleton}
{
    name=\glslink{singleton}{singleton},
    text=singleton,
    sort=singleton,
    description={Il singleton è un design pattern individuato dalla \textit{Gang of Four} che ha lo scopo di garantire che venga creata una sola istanza di una determinata classe, e di fornire un punto di accesso globale a tale istanza.
     Nel progetto questo pattern viene implementato sfruttando i moduli CommonJS, creando l'istanza di un oggetto, per poi esportarla come modulo. In questo modo l'oggetto viene creato solo una volta e risulta accessibile a tutta l'applicazione in quanto è un normale modulo CommonJS.
    }
}

\newglossaryentry{popover}
{
    name=\glslink{popover}{popover},
    text=popover,
    sort=popover,
    description={Un popover è un componente delle interfacce grafiche simile ad un pop-up, che compare quanto l'utente seleziona un elemento. A differenza di un pop-up che compare al centro dello schermo, un popover compare vicino al pulsante che l'ha reso visibile ed è collegato ad esse mediante una freccia}
}

\newglossaryentry{API}
{
    name=\glslink{API}{API},
    text=API,
    sort=API,
    description={Indica ogni insieme di procedure disponibili al programmatore, di solito raggruppate a formare un set di strumenti specifici per l'espletamento di un determinato compito all'interno di un certo programma. La finalità è ottenere un'astrazione, di solito tra l'hardware e il programmatore o tra software a basso e ad alto livello semplificando così il lavoro di programmazione}
}

\newglossaryentry{V8}
{
    name=\glslink{V8}{V8},
    text=V8,
    sort=V8,
    description={V8 è un motore JavaScript open source sviluppato da Google, attualmente incluso in Google Chrome}
}

\newglossaryentry{JavaScriptCore}
{
    name=\glslink{JavaScriptCore}{JavaScriptCore},
    text=JavaScriptCore,
    sort=JavaScriptCore,
    description={JavaScriptCore è un motore JavaScript open source sviluppato da Apple, attualmente incluso in Safari e Safari Mobile}
}

\newglossaryentry{REST}
{
    name=\glslink{REST}{REST},
    text=REST,
    sort=REST,
    description={Riferisce ad un insieme di principi di architetture di rete, i quali delineano come le risorse sono definite e indirizzate. Il termine è spesso usato nel senso di descrivere ogni semplice interfaccia che trasmette dati su HTTP}
}

\newglossaryentry{SDK}
{
name=\glslink{SDK}{SDK},
text=SDK,
sort=SDK,
description={Acronimo di \textit{Software Development Kit}, insieme di strumenti per lo sviluppo e la documentazione di software}
}
