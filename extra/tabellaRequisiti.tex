\subsection{Requisiti Funzionali}
\normalsize
\begin{longtable}{|c|>{\centering}m{7cm}|c|}
\hline
\textbf{Id Requisito} & \textbf{Descrizione} & \textbf{Stato}\\
\hline
\endhead
\hypertarget{RFO1}{RFO1} & Un utente può creare un nuovo progetto & \textcolor{Green}{\textit{Soddisfatto}}\\ \hline
\hypertarget{RFO1.1}{RFO1.1} & Un utente può scegliere un nome per un nuovo progetto & \textcolor{Green}{\textit{Soddisfatto}}\\ \hline
\hypertarget{RFO1.2}{RFO1.2} & Un utente può confermare la creazione di un nuovo progetto & \textcolor{Green}{\textit{Soddisfatto}}\\ \hline
\hypertarget{RFO1.2.1}{RFO1.2.1} & Deve essere creata la mappa mentale vuota nel progetto & \textcolor{Green}{\textit{Soddisfatto}}\\ \hline
\hypertarget{RFO1.2.1.1}{RFO1.2.1.1} & Deve essere creato un percorso di presentazione di default che include il nodo radice & \textcolor{Green}{\textit{Soddisfatto}}\\ \hline
\hypertarget{RFO2}{RFO2} & Un utente può creare un progetto da desktop & \textcolor{Green}{\textit{Soddisfatto}}\\ \hline
\hypertarget{RFD3}{RFD3} & Un utente può creare un progetto da mobile & \textcolor{Red}{\textit{Non Soddisfatto}}\\ \hline
\hypertarget{RFO4}{RFO4} & Un utente può modificare un progetto & \textcolor{Green}{\textit{Soddisfatto}}\\ \hline
\hypertarget{RFD4.1}{RFD4.1} & Un utente può modificare il nome del progetto & \textcolor{Green}{\textit{Soddisfatto}}\\ \hline
\hypertarget{RFO4.2}{RFO4.2} & Un utente può modificare la struttura della mappa & \textcolor{Green}{\textit{Soddisfatto}}\\ \hline
\hypertarget{RFO4.2.1}{RFO4.2.1} & Un utente può selezionare un nodo della mappa & \textcolor{Green}{\textit{Soddisfatto}}\\ \hline
\hypertarget{RFO4.2.2}{RFO4.2.2} & Un utente può aggiungere un nodo alla mappa & \textcolor{Green}{\textit{Soddisfatto}}\\ \hline
\hypertarget{RFO4.2.3}{RFO4.2.3} & Un utente può modificare un nodo della mappa & \textcolor{Green}{\textit{Soddisfatto}}\\ \hline
\hypertarget{RFO4.2.3.1}{RFO4.2.3.1} & Un utente può selezionare un contenuto testuale & \textcolor{Green}{\textit{Soddisfatto}}\\ \hline
\hypertarget{RFO4.2.3.2}{RFO4.2.3.2} & Un utente può selezionare un'immagine & \textcolor{Green}{\textit{Soddisfatto}}\\ \hline
\hypertarget{RFF4.2.3.3}{RFF4.2.3.3} & Un utente può selezionare un video & \textcolor{Red}{\textit{Non Soddisfatto}}\\ \hline
\hypertarget{RFO4.2.3.4}{RFO4.2.3.4} & Un utente può aggiungere un elemento testuale & \textcolor{Green}{\textit{Soddisfatto}}\\ \hline
\hypertarget{RFO4.2.3.5}{RFO4.2.3.5} & Un utente può aggiungere un’immagine & \textcolor{Green}{\textit{Soddisfatto}}\\ \hline
\hypertarget{RFO4.2.3.5.1}{RFO4.2.3.5.1} & Un utente può confermare il caricamento dell'immagine & \textcolor{Green}{\textit{Soddisfatto}}\\ \hline
\hypertarget{RFO4.2.3.5.2}{RFO4.2.3.5.2} & Un utente può scegliere l'immagine da caricare & \textcolor{Green}{\textit{Soddisfatto}}\\ \hline
\hypertarget{RFF4.2.3.6}{RFF4.2.3.6} & Un utente può aggiungere un video & \textcolor{Red}{\textit{Non Soddisfatto}}\\ \hline
\hypertarget{RFF4.2.3.6.1}{RFF4.2.3.6.1} & Un utente può scegliere la cartella contenente il video & \textcolor{Red}{\textit{Non Soddisfatto}}\\ \hline
\hypertarget{RFF4.2.3.6.2}{RFF4.2.3.6.2} & Un utente può scegliere il video da caricare & \textcolor{Red}{\textit{Non Soddisfatto}}\\ \hline
\hypertarget{RFF4.2.3.6.3}{RFF4.2.3.6.3} & Un utente può confermare il caricamento del video & \textcolor{Red}{\textit{Non Soddisfatto}}\\ \hline
\hypertarget{RFO4.2.3.7}{RFO4.2.3.7} & Un utente può modificare un elemento testuale & \textcolor{Green}{\textit{Soddisfatto}}\\ \hline
\hypertarget{RFD4.2.3.8}{RFD4.2.3.8} & Un utente può scegliere il formato di un elemento testuale & \textcolor{Red}{\textit{Non Soddisfatto}}\\ \hline
\hypertarget{RFD4.2.3.8.1}{RFD4.2.3.8.1} & Un utente può scegliere il font di un elemento testuale & \textcolor{Red}{\textit{Non Soddisfatto}}\\ \hline
\hypertarget{RFD4.2.3.8.2}{RFD4.2.3.8.2} & Un utente può scegliere il colore di un elemento testuale & \textcolor{Red}{\textit{Non Soddisfatto}}\\ \hline
\hypertarget{RFD4.2.3.9}{RFD4.2.3.9} & Un utente può spostare un elemento testuale & \textcolor{Green}{\textit{Soddisfatto}}\\ \hline
\hypertarget{RFD4.2.3.10}{RFD4.2.3.10} & Un utente può spostare un’immagine & \textcolor{Green}{\textit{Soddisfatto}}\\ \hline
\hypertarget{RFF4.2.3.11}{RFF4.2.3.11} & Un utente può spostare un video & \textcolor{Red}{\textit{Non Soddisfatto}}\\ \hline
\hypertarget{RFO4.2.3.12}{RFO4.2.3.12} & Un utente può eliminare un elemento testuale & \textcolor{Green}{\textit{Soddisfatto}}\\ \hline
\hypertarget{RFO4.2.3.13}{RFO4.2.3.13} & Un utente può eliminare un’immagine & \textcolor{Green}{\textit{Soddisfatto}}\\ \hline
\hypertarget{RFF4.2.3.14}{RFF4.2.3.14} & Un utente può eliminare un video & \textcolor{Red}{\textit{Non Soddisfatto}}\\ \hline
\hypertarget{RFD4.2.3.15}{RFD4.2.3.15} & Un utente può dare un titolo ad un nodo & \textcolor{Green}{\textit{Soddisfatto}}\\ \hline
\hypertarget{RFD4.2.3.16}{RFD4.2.3.16} & Un utente può modificare il titolo di un nodo & \textcolor{Green}{\textit{Soddisfatto}}\\ \hline
\hypertarget{RFD4.2.3.17}{RFD4.2.3.17} & Un utente può selezionare il titolo di un nodo & \textcolor{Green}{\textit{Soddisfatto}}\\ \hline
\hypertarget{RFD4.2.3.18}{RFD4.2.3.18} & Un utente può ridimensionare il titolo di un nodo & \textcolor{Green}{\textit{Soddisfatto}}\\ \hline
\hypertarget{RFD4.2.3.19}{RFD4.2.3.19} & Un utente può ridimensionare un elemento testuale presente in un nodo
& \textcolor{Green}{\textit{Soddisfatto}}\\ \hline
\hypertarget{RFD4.2.3.20}{RFD4.2.3.20} & Un utente può ridimensionare un’immagine presente in un nodo & \textcolor{Green}{\textit{Soddisfatto}}\\ \hline
\hypertarget{RFD4.2.4}{RFD4.2.4} & Un utente può spostare un nodo nella gerarchia della mappa & \textcolor{Red}{\textit{Non Soddisfatto}}\\ \hline
\hypertarget{RFO4.2.5}{RFO4.2.5} & Un utente può eliminare un nodo della mappa & \textcolor{Green}{\textit{Soddisfatto}}\\ \hline
\hypertarget{RFO4.2.6}{RFO4.2.6} & Un utente può creare una associazione tra nodi & \textcolor{Green}{\textit{Soddisfatto}}\\ \hline
\hypertarget{RFO4.2.7}{RFO4.2.7} & Un utente può eliminare una associazione tra nodi & \textcolor{Green}{\textit{Soddisfatto}}\\ \hline
\hypertarget{RFO4.2.8}{RFO4.2.8} & Un utente può selezionare un'associazione tra due nodi della mappa & \textcolor{Green}{\textit{Soddisfatto}}\\ \hline
\hypertarget{RFD4.2.9}{RFD4.2.9} & Un utente può modificare la posizione nella quale viene disegnato un nodo & \textcolor{Red}{\textit{Non Soddisfatto}}\\ \hline
\hypertarget{RFD4.3}{RFD4.3} & Un utente può creare un percorso di presentazione personalizzato & \textcolor{Green}{\textit{Soddisfatto}}\\ \hline
\hypertarget{RFD4.3.1}{RFD4.3.1} & Un utente può scegliere un nome per un percorso di presentazione personalizzato & \textcolor{Green}{\textit{Soddisfatto}}\\ \hline
\hypertarget{RFD4.3.2}{RFD4.3.2} & Un utente può scegliere un nodo di una mappa mentale come primo frame di un percorso di presentazione personalizzato & \textcolor{Green}{\textit{Soddisfatto}}\\ \hline
\hypertarget{RFD4.3.3}{RFD4.3.3} & Un utente può confermare la creazione di un percorso di presentazione personalizzato & \textcolor{Green}{\textit{Soddisfatto}}\\ \hline
\hypertarget{RFD4.4}{RFD4.4} & Un utente può modificare un percorso di presentazione personalizzato & \textcolor{Green}{\textit{Soddisfatto}}\\ \hline
\hypertarget{RFD4.4.1}{RFD4.4.1} & Un utente può aggiungere un frame al percorso di presentazione personalizzato & \textcolor{Green}{\textit{Soddisfatto}}\\ \hline
\hypertarget{RFD4.4.2}{RFD4.4.2} & Un utente può spostare un frame all'interno di un percorso di presentazione personalizzato & \textcolor{Red}{\textit{Non Soddisfatto}}\\ \hline
\hypertarget{RFD4.4.3}{RFD4.4.3} & Un utente può eliminare un frame da un percorso di presentazione personalizzato & \textcolor{Green}{\textit{Soddisfatto}}\\ \hline
\hypertarget{RFD4.4.4}{RFD4.4.4} & Un utente può marcare un percorso di presentazione come il percorso di presentazione principale di un progetto & \textcolor{Red}{\textit{Non Soddisfatto}}\\ \hline
\hypertarget{RFD4.4.5}{RFD4.4.5} & Un utente può modificare il nome di un percorso di presentazione personalizzato & \textcolor{Green}{\textit{Soddisfatto}}\\ \hline
\hypertarget{RFD4.5}{RFD4.5} & Un utente può eliminare un percorso di presentazione personalizzato & \textcolor{Green}{\textit{Soddisfatto}}\\ \hline
\hypertarget{RFD4.5.1}{RFD4.5.1} & Un utente può confermare l'eliminazione del percorso di presentazione selezionato & \textcolor{Green}{\textit{Soddisfatto}}\\ \hline
\hypertarget{RFF4.6}{RFF4.6} & Un utente può scegliere le impostazioni generali dell’aspetto grafico del progetto & \textcolor{Green}{\textit{Soddisfatto}}\\ \hline
\hypertarget{RFF4.6.1}{RFF4.6.1} & Un utente può scegliere il formato di default per il testo & \textcolor{Green}{\textit{Soddisfatto}}\\ \hline
\hypertarget{RFF4.6.1.1}{RFF4.6.1.1} & Un utente può scegliere una famiglia di font di default per il testo & \textcolor{Green}{\textit{Soddisfatto}}\\ \hline
\hypertarget{RFF4.6.1.2}{RFF4.6.1.2} & Un utente può scegliere un colore di default per il testo & \textcolor{Green}{\textit{Soddisfatto}}\\ \hline
\hypertarget{RFF4.6.2}{RFF4.6.2} & Un utente può scegliere un colore di sfondo per i frame del progetto & \textcolor{Green}{\textit{Soddisfatto}}\\ \hline
\hypertarget{RFF4.6.3}{RFF4.6.3} & Un utente può confermare le impostazioni scelte & \textcolor{Green}{\textit{Soddisfatto}}\\ \hline
\hypertarget{RFD4.7}{RFD4.7} & Un utente può selezionare un percorso di presentazione personalizzato & \textcolor{Green}{\textit{Soddisfatto}}\\ \hline
\hypertarget{RFO5}{RFO5} & Un utente può modificare un progetto da desktop & \textcolor{Green}{\textit{Soddisfatto}}\\ \hline
\hypertarget{RFD6}{RFD6} & Un utente può modificare un progetto da mobile & \textcolor{Red}{\textit{Non Soddisfatto}}\\ \hline
\hypertarget{RFO7}{RFO7} & Un utente può eseguire una presentazione & \textcolor{Green}{\textit{Soddisfatto}}\\ \hline
\hypertarget{RFD7.1}{RFD7.1} & Un utente può scegliere un percorso di presentazione relativo ad un progetto & \textcolor{Green}{\textit{Soddisfatto}}\\ \hline
\hypertarget{RFO7.2}{RFO7.2} & Un utente a partire da un frame può scegliere di spostarsi al frame successivo della sequenza di presentazione & \textcolor{Green}{\textit{Soddisfatto}}\\ \hline
\hypertarget{RFO7.3}{RFO7.3} & Un utente a partire da un frame può scegliere di spostarsi al frame precedente della sequenza di presentazione & \textcolor{Green}{\textit{Soddisfatto}}\\ \hline
\hypertarget{RFD7.4}{RFD7.4} & Un utente, a partire da un frame, può decidere di spostarsi ad un qualsiasi altro frame presente nella mappa mentale & \textcolor{Red}{\textit{Non Soddisfatto}}\\ \hline
\hypertarget{RFD7.4.1}{RFD7.4.1} & Un utente può visualizzare tutti i frame presenti nella mappa mentale & \textcolor{Red}{\textit{Non Soddisfatto}}\\ \hline
\hypertarget{RFD7.4.2}{RFD7.4.2} & Un utente può selezionare un frame da visualizzare, in modo da poter effettuare una presentazione non lineare & \textcolor{Red}{\textit{Non Soddisfatto}}\\ \hline
\hypertarget{RFD7.5}{RFD7.5} & Un utente può chiudere una presentazione & \textcolor{Green}{\textit{Soddisfatto}}\\ \hline
\hypertarget{RFD7.6}{RFD7.6} & Un utente può passare ad un frame correlato al frame che sta visualizzando, dove per correlato si intende che tra i due nodi della mappa mentale c’è un’associazione padre-figlio oppure è stata creata un’associazione da parte dell’utente & \textcolor{Green}{\textit{Soddisfatto}}\\ \hline
\hypertarget{RFD7.6.1}{RFD7.6.1} & Un utente può visualizzare tutti i frame correlati al frame che sta visualizzando & \textcolor{Green}{\textit{Soddisfatto}}\\ \hline
\hypertarget{RFD7.6.2}{RFD7.6.2} & Un utente può selezionare un frame tra quelli mostrati dal sistema & \textcolor{Green}{\textit{Soddisfatto}}\\ \hline
\hypertarget{RFO7.7}{RFO7.7} & L'utente può vedere il contenuto del frame presente in un nodo & \textcolor{Green}{\textit{Soddisfatto}}\\ \hline
\hypertarget{RFO8}{RFO8} & Un utente può eseguire una presentazione da desktop & \textcolor{Green}{\textit{Soddisfatto}}\\ \hline
\hypertarget{RFO9}{RFO9} & Un utente può eseguire una presentazione da mobile & \textcolor{Green}{\textit{Soddisfatto}}\\ \hline
\hypertarget{RFO10}{RFO10} & Un utente può eliminare un progetto precedentemente creato & \textcolor{Green}{\textit{Soddisfatto}}\\ \hline
\hypertarget{RFO10.1}{RFO10.1} & Un utente può selezionare un progetto da eliminare & \textcolor{Green}{\textit{Soddisfatto}}\\ \hline
\hypertarget{RFO10.2}{RFO10.2} & Un utente può confermare l’eliminazione di uno dei suoi progetti & \textcolor{Green}{\textit{Soddisfatto}}\\ \hline
\hypertarget{RFO11}{RFO11} & Un utente può aprire un progetto & \textcolor{Green}{\textit{Soddisfatto}}\\ \hline
\hypertarget{RFO11.1}{RFO11.1} & Un utente può confermare l'apertura di un progetto & \textcolor{Green}{\textit{Soddisfatto}}\\ \hline
\hypertarget{RFO11.2}{RFO11.2} & Un utente può scegliere un progetto da aprire & \textcolor{Green}{\textit{Soddisfatto}}\\ \hline
\hypertarget{RFO11.3}{RFO11.3} & Un utente può visualizzare tutti i progetti che ha creato & \textcolor{Green}{\textit{Soddisfatto}}\\ \hline
\hypertarget{RFD12}{RFD12} & Un utente può esportare tutte le presentazioni & \textcolor{Red}{\textit{Non Soddisfatto}}\\ \hline
\hypertarget{RFD12.1}{RFD12.1} & Un utente può scegliere una cartella in cui salvare una pagina web con tutte le presentazioni di un progetto & \textcolor{Red}{\textit{Non Soddisfatto}}\\ \hline
\hypertarget{RFD12.2}{RFD12.2} & Un utente può scegliere un nome per una pagina web con tutte le presentazioni di un progetto & \textcolor{Red}{\textit{Non Soddisfatto}}\\ \hline
\hypertarget{RFD12.3}{RFD12.3} & Un utente può confermare l'esportazione di una pagina web con tutte le presentazioni di un progetto & \textcolor{Red}{\textit{Non Soddisfatto}}\\ \hline
\hypertarget{RFF13}{RFF13} & Un utente può esportare una mappa mentale in pdf & \textcolor{Green}{\textit{Soddisfatto}}\\ \hline
\hypertarget{RFF13.1}{RFF13.1} & Un utente può scegliere la cartella in cui esportare una mappa mentale in pdf & \textcolor{Green}{\textit{Soddisfatto}}\\ \hline
\hypertarget{RFF13.2}{RFF13.2} & Un utente può scegliere un nome per il file pdf di una mappa mentale & \textcolor{Green}{\textit{Soddisfatto}}\\ \hline
\hypertarget{RFF13.3}{RFF13.3} & Un utente può confermare l'esportazione in pdf di una mappa mentale & \textcolor{Green}{\textit{Soddisfatto}}\\ \hline
\hypertarget{RFD14}{RFD14} & Un utente può esportare la presentazione che sta visualizzando in pdf & \textcolor{Green}{\textit{Soddisfatto}}\\ \hline
\hypertarget{RFD14.1}{RFD14.1} & Un utente può confermare l'esportazione in pdf & \textcolor{Green}{\textit{Soddisfatto}}\\ \hline
\hypertarget{RFD14.2}{RFD14.2} & Un utente può scegliere la cartella in cui salvare una presentazione in pdf & \textcolor{Green}{\textit{Soddisfatto}}\\ \hline
\hypertarget{RFD14.3}{RFD14.3} & Un utente può scegliere un nome per la presentazione & \textcolor{Green}{\textit{Soddisfatto}}\\ \hline
\hypertarget{RFD14.4}{RFD14.4} & Un utente può scegliere un percorso di presentazione relativo ad un progetto & \textcolor{Red}{\textit{Non Soddisfatto}}\\ \hline
\hypertarget{RFD15}{RFD15} & Un utente può stampare una mappa mentale & \textcolor{Green}{\textit{Soddisfatto}}\\ \hline
\hypertarget{RFF15.1}{RFF15.1} & Un utente può visualizzare l'anteprima di stampa di una mappa mentale & \textcolor{Green}{\textit{Soddisfatto}}\\ \hline
\hypertarget{RFF15.2}{RFF15.2} & Un utente può scegliere le impostazioni di pagina per la stampa di una mappa mentale & \textcolor{Green}{\textit{Soddisfatto}}\\ \hline
\hypertarget{RFD15.3}{RFD15.3} & Un utente può confermare la stampa di una mappa mentale & \textcolor{Green}{\textit{Soddisfatto}}\\ \hline
\hypertarget{RFO16}{RFO16} & Un utente può stampare una presentazione che sta visualizzando & \textcolor{Green}{\textit{Soddisfatto}}\\ \hline
\hypertarget{RFO16.1}{RFO16.1} & Un utente può confermare la stampa di una presentazione & \textcolor{Green}{\textit{Soddisfatto}}\\ \hline
\hypertarget{RFF16.2}{RFF16.2} & Un utente può visualizzare l'anteprima di stampa di una presentazione & \textcolor{Green}{\textit{Soddisfatto}}\\ \hline
\hypertarget{RFF16.3}{RFF16.3} & Un utente può scegliere le impostazioni di pagina per la stampa di una presentazione & \textcolor{Green}{\textit{Soddisfatto}}\\ \hline
\hypertarget{RFF16.3.1}{RFF16.3.1} & L’utente può scegliere quanti frame stampare su una pagina & \textcolor{Red}{\textit{Non Soddisfatto}}\\ \hline
\hypertarget{RFD16.4}{RFD16.4} & Un utente può scegliere un percorso di presentazione relativo ad un progetto & \textcolor{Red}{\textit{Non Soddisfatto}}\\ \hline
\hypertarget{RFD17}{RFD17} & Un utente può eseguire una presentazione non lineare & \textcolor{Green}{\textit{Soddisfatto}}\\ \hline
\hypertarget{RFO18}{RFO18} & L'applicazione fornisce un effetto grafico di default per la transizione tra nodi in un percorso di default & \textcolor{Green}{\textit{Soddisfatto}}\\ \hline
\hypertarget{RFD19}{RFD19} &  L'utente può scegliere, tra un set predefinito, l'effetto grafico da applicare alla transizione tra nodi & \textcolor{Red}{\textit{Non Soddisfatto}}\\ \hline
\hypertarget{RFD19.1}{RFD19.1} & Un utente può inserire l'effetto grafico zoom in una transizione tra nodi & \textcolor{Red}{\textit{Non Soddisfatto}}\\ \hline
\hypertarget{RFD20}{RFD20} & Un utente può inserire l'effetto grafico clip in una presentazione & \textcolor{Red}{\textit{Non Soddisfatto}}\\ \hline
\hypertarget{RFF21}{RFF21} & Un utente può realizzare una presentazione in collaborazione con altri utenti & \textcolor{Red}{\textit{Non Soddisfatto}}\\ \hline
\hypertarget{RFD22}{RFD22} & Un utente può creare una mappa mentale & \textcolor{Green}{\textit{Soddisfatto}}\\ \hline
\hypertarget{RFD23}{RFD23} & L'utente viene supportato nella fase creativa & \textcolor{Red}{\textit{Non Soddisfatto}}\\ \hline
\hypertarget{RFD24}{RFD24} & Un utente può creare una presentazione usando la tecnica dello storytelling & \textcolor{Red}{\textit{Non Soddisfatto}}\\ \hline
\hypertarget{RFF25}{RFF25} & L'utente può chiudere il progetto correntemente caricato nel sistema & \textcolor{Green}{\textit{Soddisfatto}}\\ \hline
\hypertarget{RFF25.1}{RFF25.1} & Il sistema deve poter permettere all'utente di salvare le modifiche al progetto corrente prima di chiuderlo & \textcolor{Green}{\textit{Soddisfatto}}\\ \hline
\hypertarget{RFD26}{RFD26} & L'utente può consultare il manuale direttamente dall'applicazione & \textcolor{Green}{\textit{Soddisfatto}}\\ \hline
\hypertarget{RFO27}{RFO27} & Il programma deve fornire un'interfaccia utente & \textcolor{Green}{\textit{Soddisfatto}}\\ \hline
\hypertarget{RFF28}{RFF28} & Un utente può annullare l'ultima modifica effettuata & \textcolor{Red}{\textit{Non Soddisfatto}}\\ \hline
\hypertarget{RFF29}{RFF29} & Un utente può ripristinare l'ultima modifica annullata & \textcolor{Red}{\textit{Non Soddisfatto}}\\ \hline
\hypertarget{RFO30}{RFO30} & Un utente può scegliere tra registrazione e login & \textcolor{Green}{\textit{Soddisfatto}}\\ \hline
\hypertarget{RFO30.1}{RFO30.1} & Un utente può creare un account personale & \textcolor{Green}{\textit{Soddisfatto}}\\ \hline
\hypertarget{RFO30.1.1}{RFO30.1.1} & Un utente può scegliere una mail per un account personale & \textcolor{Green}{\textit{Soddisfatto}}\\ \hline
\hypertarget{RFO30.1.2}{RFO30.1.2} & Un utente può scegliere una password per un account personale & \textcolor{Green}{\textit{Soddisfatto}}\\ \hline
\hypertarget{RFO30.1.3}{RFO30.1.3} & Un utente può confermare la creazione di un nuovo account personale & \textcolor{Green}{\textit{Soddisfatto}}\\ \hline
\hypertarget{RFO30.2}{RFO30.2} & Un utente può accedere al suo account personale & \textcolor{Green}{\textit{Soddisfatto}}\\ \hline
\hypertarget{RFO30.2.1}{RFO30.2.1} & Un utente può inserire l’indirizzo email con il quale si è registrato & \textcolor{Green}{\textit{Soddisfatto}}\\ \hline
\hypertarget{RFO30.2.2}{RFO30.2.2} & Un utente può inserire la password relativa al proprio account nel sistema & \textcolor{Green}{\textit{Soddisfatto}}\\ \hline
\hypertarget{RFO30.2.3}{RFO30.2.3} & Un utente può confermare i dati inseriti per accedere al proprio account & \textcolor{Green}{\textit{Soddisfatto}}\\ \hline
\hypertarget{RFO30.3}{RFO30.3} & Il sistema può visualizzare un messaggio di errore, in caso di errato inserimento dei dati da parte dell’utente, al momento della registrazione & \textcolor{Green}{\textit{Soddisfatto}}\\ \hline
\hypertarget{RFO30.4}{RFO30.4} & Il sistema può visualizzare un messaggio di errore, in caso di errato inserimento dei dati da parte dell’utente, al momento dell’autenticazione & \textcolor{Green}{\textit{Soddisfatto}}\\ \hline
\hypertarget{RFD31}{RFD31} & Un utente può gestire i suoi dati personali & \textcolor{Red}{\textit{Non Soddisfatto}}\\ \hline
\hypertarget{RFD31.1}{RFD31.1} & Un utente può modificare la propria password & \textcolor{Red}{\textit{Non Soddisfatto}}\\ \hline
\hypertarget{RFD31.1.1}{RFD31.1.1} & Un utente può inserire la vecchia password & \textcolor{Red}{\textit{Non Soddisfatto}}\\ \hline
\hypertarget{RFD31.1.2}{RFD31.1.2} & Un utente può inserire una nuova password & \textcolor{Red}{\textit{Non Soddisfatto}}\\ \hline
\hypertarget{RFD31.1.3}{RFD31.1.3} & Un utente può confermare la modifica della propria password & \textcolor{Red}{\textit{Non Soddisfatto}}\\ \hline
\hypertarget{RFF31.2}{RFF31.2} & Un utente può cancellare il proprio account & \textcolor{Red}{\textit{Non Soddisfatto}}\\ \hline
\hypertarget{RFF31.2.1}{RFF31.2.1} & Un utente può confermare la cancellazione del proprio account & \textcolor{Red}{\textit{Non Soddisfatto}}\\ \hline
\hypertarget{RFO32}{RFO32} & Un utente può effettuare il logout & \textcolor{Green}{\textit{Soddisfatto}}\\ \hline
\hypertarget{RFD33}{RFD33} & Il sistema deve notificare all’utente che è già presente un altro progetto con lo stesso nome & \textcolor{Green}{\textit{Soddisfatto}}\\ \hline
\hypertarget{RFD34}{RFD34} & Un utente può eseguire una presentazione offline e senza dover autenticarsi & \textcolor{Red}{\textit{Non Soddisfatto}}\\ \hline
\hypertarget{RFD34.1}{RFD34.1} & Senza essersi autenticato e senza una connessione ad internet, l’utente può passare ad un frame correlato al frame che sta visualizzando & \textcolor{Red}{\textit{Non Soddisfatto}}\\ \hline
\hypertarget{RFD34.1.1}{RFD34.1.1} & Senza essersi autenticato e senza una connessione ad internet, l’utente può visualizzare tutti i frame correlati al frame che sta visualizzando & \textcolor{Red}{\textit{Non Soddisfatto}}\\ \hline
\hypertarget{RFD34.1.2}{RFD34.1.2} & Senza essersi autenticato e senza una connessione ad internet, l’utente può selezionare un frame tra quelli mostrati dal sistema & \textcolor{Red}{\textit{Non Soddisfatto}}\\ \hline
\hypertarget{RFD34.2}{RFD34.2} & Senza essersi autenticato e senza una connessione ad internet, l’utente, a partire da un frame, può scegliere di spostarsi al frame successivo della sequenza di presentazione & \textcolor{Red}{\textit{Non Soddisfatto}}\\ \hline
\hypertarget{RFD34.3}{RFD34.3} & Senza essersi autenticato e senza una connessione ad internet, l’utente, a partire da un frame, può scegliere di spostarsi al frame precedente della sequenza di presentazione & \textcolor{Red}{\textit{Non Soddisfatto}}\\ \hline
\hypertarget{RFD34.4}{RFD34.4} & Senza essersi autenticato e senza una connessione ad internet, l’utente, a partire da un frame, può decidere di spostarsi ad un qualsiasi altro frame presente nella mappa mentale & \textcolor{Red}{\textit{Non Soddisfatto}}\\ \hline
\hypertarget{RFD34.4.1}{RFD34.4.1} & Senza essersi autenticato e senza una connessione ad internet, l’utente può visualizzare tutti i frame presenti nella mappa mentale & \textcolor{Red}{\textit{Non Soddisfatto}}\\ \hline
\hypertarget{RFD34.4.2}{RFD34.4.2} & Senza essersi autenticato e senza una connessione ad internet, l’utente può selezionare un frame da visualizzare, in modo da poter effettuare una presentazione non lineare & \textcolor{Red}{\textit{Non Soddisfatto}}\\ \hline
\hypertarget{RFD34.5}{RFD34.5} & Senza essersi autenticato e senza una connessione ad internet, l’utente può chiudere una presentazione & \textcolor{Red}{\textit{Non Soddisfatto}}\\ \hline
\hypertarget{RFD35}{RFD35} & Il sistema deve comunicare all'utente gli errori di comunicazione tra la componente Back-End e la componente Front-End & \textcolor{Green}{\textit{Soddisfatto}}\\ \hline
\hypertarget{RFO36}{RFO36} & Il server deve poter essere configurato ed avviato per fornire le funzionalità di Back-End & \textcolor{Green}{\textit{Soddisfatto}}\\ \hline
\caption[Requisiti Funzionali]{Requisiti Funzionali}
\label{tabella:req0}
\end{longtable}
\clearpage
\subsection{Requisiti di Qualità}
\normalsize
\begin{longtable}{|c|>{\centering}m{7cm}|c|}
\hline
\textbf{Id Requisito} & \textbf{Descrizione} & \textbf{Stato}\\
\hline
\endhead
\hypertarget{RQO1}{RQO1} & Deve essere fornito un manuale utente & \textcolor{Green}{\textit{Soddisfatto}}\\ \hline
\hypertarget{RQO1.1}{RQO1.1} & Il manuale utente deve contenere una sezione in cui viene spiegato come installare correttamente l'applicazione & \textcolor{Green}{\textit{Soddisfatto}}\\ \hline
\hypertarget{RQO1.2}{RQO1.2} & Il manuale utente deve contenere una sezione in cui viene approfonditamente spiegato come utilizzare l'applicazione & \textcolor{Green}{\textit{Soddisfatto}}\\ \hline
\hypertarget{RQO1.3}{RQO1.3} & Il manuale utente deve includere una sezione contenente un elenco di possibili errori e malfunzionamenti dell'applicazione e le loro possibili cause & \textcolor{Green}{\textit{Soddisfatto}}\\ \hline
\hypertarget{RQO1.4}{RQO1.4} & Il manuale utente deve contenere una sezione che spiega come segnalare eventuali errori e malfunzionamenti & \textcolor{Green}{\textit{Soddisfatto}}\\ \hline
\hypertarget{RQD2}{RQD2} & Deve essere fornito un manuale per gli utenti sviluppatori che intendono estendere l'applicazione & \textcolor{Red}{\textit{Non Soddisfatto}}\\ \hline
\hypertarget{RQD2.1}{RQD2.1} & Il manuale per gli utenti sviluppatori che intendono estendere l'applicazione deve contenere una sezione che spiega come segnalare eventuali errori o malfunzionamenti & \textcolor{Red}{\textit{Non Soddisfatto}}\\ \hline
\hypertarget{RQF3}{RQF3} & La documentazione per l'utente deve essere disponibile in lingua inglese & \textcolor{Red}{\textit{Non Soddisfatto}}\\ \hline
\hypertarget{RQO4}{RQO4} & La documentazione per l'utente deve essere disponibile in lingua italiana & \textcolor{Green}{\textit{Soddisfatto}}\\ \hline
\hypertarget{RQO5}{RQO5} & La documentazione del codice sorgente del software deve essere prodotta utilizzando \textit{JSDoc} & \textcolor{Green}{\textit{Soddisfatto}}\\ \hline
\hypertarget{RQF6}{RQF6} & La documentazione del codice sorgente del software deve essere prodotta in lingua inglese utilizzando \textit{JSDoc} & \textcolor{Red}{\textit{Non Soddisfatto}}\\ \hline
\caption[Requisiti di Qualità]{Requisiti di Qualità}
\label{tabella:req2}
\end{longtable}
\clearpage

\begin{longtable}{|c|c|}
\caption{Average Scalar Flux Groups 1 and 2 Test  C}
\label{tlabel} \\
\hline
\rowcolor[gray]{0.85}\textbf{Average Scalar Flux Group 1} &\textbf{Average Scalar Flux Group 2}  \\  
\hline
2.185692E-17    &   4.194769E-18       \\ \hline
-1.202652E-04   &   1.202652E-04 \\ \hline
-2.405521E-04   &   2.405521E-04 \\ \hline
\end{longtable}
