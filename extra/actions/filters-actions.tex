%\subsubsection{FiltersActions}

Questo modulo contiene i metodi che creano le azioni riguardanti il sistema di gestione dei filtri.

L'oggetto esposto dal modulo contiene i seguenti metodi:
\begin{itemize}
\item \textbf{loadFilterItems(filter, appliedFilters)}

Il metodo recupera la lista dei possibili valori dell'oggetto \texttt{filter} ricevuto come parametor a partire dall'URL presente nel campo dati \texttt{filter.url}. 
Alla richiesta dei dati viene aggiunta l'informazione riguardante i filtri che sono correntemente applicati, in quando la lista dei valori che può assumere un filtro dipende dai filtri che sono stati applicati.

Una volta ottenuti i dati richiede il \textit{dispatch} dell'oggetto:
\begin{lstlisting}[language=JSON, caption=Action - load filter items]
{
  "actionType": string, //FiltersConstants.LOAD_FILTER_ITEMS
  "filterItems: Array<FilterItem> //Array con i possibili valori
}
\end{lstlisting}

Quando \texttt{FiltersStore} riceve questo oggetto, deve caricare al suo interno i dati contenuti nell'oggetto in modo che questi siano disponibili all'interno dell'applicazione. Nel caso \texttt{FiltersStore} contenga già i valori di un filtro, questi devono essere sovrascritti.

\item \textbf{applyFilter(filterData)}

Il metodo richiede il \textit{dispatch} dell'oggetto:
\begin{lstlisting}[language=JSON, caption=Action - apply filter]
{
  "actionType": string, //FiltersConstants.APPLY_FILTER
  "filterData": {filter: Filter, filterItem: FilterItem} //coincide con l'oggetto ricevuto come parametro
}
\end{lstlisting}

Quando \texttt{FiltersStore} riceve questo oggetto, deve applicare il filtro contenuto nel campo dati \texttt{filterData}.

\item \textbf{removeFilter(filter)}

Il metodo richiede il \textit{dispatch} dell'oggetto:
\begin{lstlisting}[language=JSON, caption=Action - remove filter]
{
  "actionType": string, //FiltersConstants.REMOVE_FILTER
  "filter": Filter //Filtro da rimovure, coincide con l'oggetto ricevuto come parametro
}
\end{lstlisting}

Quando \texttt{FiltersStore} riceve questo oggetto, deve rimuovere dai filtri applicati il filtro contenuto nel campo dati \texttt{filter}.


\item \textbf{resetAppliedFilters()}

Il metodo richiede il \textit{dispatch} dell'oggetto:
\begin{lstlisting}[language=JSON, caption=Action - clear assets]
{
  "actionType": string //FiltersConstants.RESET_APPLIED_FILTERS
}
\end{lstlisting}

Quando \texttt{FiltersStore} riceve questo oggetto, deve cancellare tutte le informazioni relative ai filtri che sono stati applicati.

\end{itemize}