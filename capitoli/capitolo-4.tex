% !TEX encoding = UTF-8
% !TEX TS-program = pdflatex
% !TEX root = ../tesi.tex
% !TEX spellcheck = it-IT

%**************************************************************
\chapter{Valutazioni retrospettive}
\label{cap:analisi-requisiti}
%**************************************************************

\section{Bilancio sui risultati}

In questa sezione darò evidenza di aver soddisfatto tutti gli obiettivi di stage, fornendo un consuntivo di quanto realizzato, confrontato con il preventivo di inizio stage.

\section{Bilancio formativo}

In questa sezione confronterò la mia formazione all'inizio dello stage con quella ottenuta alla fine, evidenziando quanto appreso e il tipo di maturazione ottenuta sul piano tecnico e metodologico.

\section{Distanza tra contesto lavorativo e mondo accademico}

In questa sezione fornirò un GAP di conoscenze, evidenziando gli elementi nulli sul quale il corso di laurea non è riuscito a formarmi. Fornirò dunque una critica costruttiva nei confronti del mondo accademico, confrontando il mio bagaglio in ingresso con quello in uscita e presentando dei suggerimenti in ottica di miglioramento del corso di laurea.
